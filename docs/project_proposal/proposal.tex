\documentclass[11pt,letterpaper]{article}
\usepackage[utf8]{inputenc}
\usepackage[margin=1in]{geometry}
\usepackage{times}
\usepackage{setspace}
\usepackage{apacite}
\usepackage{url}

% Set line spacing
\doublespacing

% Title page
\title{Machine Learning for Real-Time Detection and Localization of Network Failures}
\author{Michael Hernandez}
\date{September 12, 2025}

\begin{document}

% Cover page
\begin{titlepage}
    \centering
    \vspace*{2in}
    {\LARGE \textbf{Machine Learning for Real-Time Detection and Localization of Network Failures}}
    
    \vspace{1in}
    {\large CUNY School of Professional Studies}
    
    \vspace{0.5in}
    {\large Michael Hernandez}
    
    \vspace{0.5in}
    {\large September 12, 2025}
    
    \vfill
\end{titlepage}

% Reset page numbering after title page
\setcounter{page}{1}

\section{Topic Description}

This project develops a pragmatic system that ingests two operational signals—BGP routing updates and device logs—and applies approachable machine-learning to detect failure-induced incidents quickly and infer where they most likely originate (e.g., top-of-rack, spine/route-reflector, or edge/provider). The emphasis is operator value: fewer noisy alerts, faster first actions, and an explanation of "what broke and where."

\section{Problem Description}

Large BGP-routed environments generate many alarms but little immediate guidance about what truly matters or where to begin. Traditional SNMP/syslog thresholds alert on hard failures yet often over-page on benign edge-local events and under-explain control-plane or egress faults, leading to manual correlation and delays. In a production setting with thousands of switches and anycast/VXLAN, this increases time to detect and time to resolve. A system that correlates control-plane churn with structured log patterns and a simple role-based topology can reduce detection delay, provide a first-guess fault location, and suppress noise that is confined to a single rack or host.

\section{Solution Description}

The system extracts simple features from BGP updates (withdrawals, AS-path churn, next-hop shifts) and log templates (counts, severity, burstiness), then applies lightweight, well-documented methods: Matrix Profile for time-series anomaly cues on BGP streams and Isolation Forest for device-level log vectors \cite{scott2024, cheng2021}. Scores are normalized and fused, then passed through a topology-aware localization step that uses the network's role map (server, ToR, spine/RR, edge) to propose a likely origin and to down-rank edge-local flaps \cite{tan2024}. A small Streamlit dashboard presents the alert, the suspected location, and the top contributing signals; the goal is earlier, clearer action rather than another raw alarm. To tie the work to operator outcomes, the evaluation compares this system to manual SNMP/syslog triage and reports detection delay, event-level precision/recall/F1, localization accuracy (Hit@k), and page reduction \cite{mohammed2021}.

\section{Research}

\subsection{Coursework Foundations}

My degree provided the \textbf{foundations} for this project while my work experience supplies domain depth.

\begin{itemize}
    \item \textbf{Python, data structures, and databases} (IS 210/211, IS 361, IS 362) support streaming parsers, clean feature extraction, and simple event/metric schemas.
    \item \textbf{Networks and infrastructure} (IS 205, IS 260) ground the failure taxonomy and the SNMP/syslog baseline.
    \item \textbf{Systems analysis, enterprise architectures, and project management} (IS 320, IS 300, PROM 210) inform requirements, a layered design (ingest → features → models → localization → UI), and a semester plan.
    \item \textbf{Security and strategy} (IS 250, IS 350) guide secure telemetry handling, RBAC for the dashboard, and value framing.
\end{itemize}

\subsection{Scholarly and Technical Resources}

I will use recent papers and accessible methods: Matrix Profile for BGP time series \cite{scott2024}, multi-scale sequence models as a supervised baseline \cite{cheng2021}, topology-aware analyses for structure \cite{tan2024}, and an operations-oriented study tying analytics to operator decisions \cite{mohammed2021}. For logs, approachable unsupervised techniques (e.g., Isolation Forest / One-Class SVM) provide simple, interpretable starting points. Clear UML and architecture diagrams will keep the narrative plain and direct.

\subsection{Development \& Evaluation Resources}

A containerized \textbf{virtual lab} (FRRouting with two spines, two ToRs, two edges, and multiple "server" peers) will generate realistic BGP updates and logs. I will inject four representative failures—\textbf{one-way loss of signal, route-reflector crash, edge/provider outage, and server crash}—then compare the system's alerts against a \textbf{baseline of manual SNMP/syslog triage}. Event-level precision, recall, and \textbf{F1} will be reported along with the \textbf{detection delay} to first alert and localization \textbf{Hit@k}; results will be summarized in a short paper, an 18-slide presentation, and a live demo.

\section{Writing \& Formatting}

The paper will use \textbf{plain, professional language} and avoid jargon where possible, defining terms when needed (e.g., "BGP updates" as "routing change messages"). In-text \textbf{APA citations} will be included with a reference list to meet formatting expectations.

% References
\bibliographystyle{apacite}
\bibliography{references}

\end{document}
